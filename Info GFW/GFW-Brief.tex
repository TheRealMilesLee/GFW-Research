\documentclass[11pt]{article}
\title{GFW: Protection or censorship?\\ \large A brief overview on China's Great Firewall system}
\author{Hengyi Li}
\newcommand{\name}{Hengyi Li}
\usepackage[paper=letterpaper, margin=1in, headheight=13.6pt]{geometry}
\usepackage{fancyhdr}
\pagestyle{fancy}
\cfoot{Page \thepage}

\usepackage[parfill]{parskip}
\usepackage{amsmath}
\usepackage{graphicx}
\usepackage{fancyvrb}
\usepackage{upquote}
\usepackage{multicol}
\usepackage{hyperref}

\begin{document}
\maketitle
\bibliographystyle{mla}
\section{Introduction}

The Internet has come a long way today. As a bridge for the flow of information,
it carries the thoughts, ideas, opinions and needs of all people. However,
because of the value attributes of information, the flow of information may
inadvertently touch the interests of certain people or groups. This may lead to
theft by criminals, or it may be a threat to national security, but in any case,
the network firewall has been built as a result. In China, there also exists
such a wall, which is not only able to defend against cyber-attacks from outside
the country, but at the same time, and most importantly, prevents the flow of
network traffic from within the country to outside the country. This is the
subject of this article - The Great Firewall.

\section{Background}
Great Firewall (hereinafter referred to as GFW) is a traffic censorship and
attack defense system based on Cisco's intrusion detection system \cite{CISCO}.
This defense system combines active and passive detection capabilities to censor
China's Internet traffic \cite{DetectBlockShadowSocks}.
As far as we know from our testing, the system is running on IPV4, but we have no way of knowing if the system is compatible with
IPV6 as well. The main reason for this is that IPV6 is not very popular in China,
and even if we could ask the carriers to enable it, they would not be very
willing to do so for security reasons, and it would require a series of very
complicated procedures to complete. What's worse, the GFW is still a black box
system for outsiders and researchers, and the existence of the system is not
recognized at the national level, and there is no official documentation about
the system. All research into the system is currently in the ``poke around and
see what happend'' phase, so most of the results and findings below come from
community organizations and web forum members. We would like to thank the
members and contributors of the gfw.repot \cite{GFWReport} website for providing a wealth of
documentation and experimental data without which we would not have been able to
gain such a deep understanding of GFW. In order to begin to understand how the
GFW network in China works, we need to first understand the composition of the
network structure in China, which has a huge difference with here in the U.S.

\section{China's internet architecture}
In China, the Internet architecture consists of layers and layers of NATs, and
most people do not have public IPs. there are only three Internet Service
Providers (ISPs) in China, China Mobile, China Unicom and China Telecom, all of
which are state-owned enterprises. All three are state-owned enterprises. All
three companies are state-owned and have a monopoly on the supply of cell phones,
TV, and Internet services in China. The reason for these three providers to use
Carrier Grade NAT is that China has too many network devices, and IPV4 addresses
are limited, coupled with a large population base, which makes China face the
problem of insufficient IPV4 addresses earlier than other countries in the world.
The use of carrier grade NAT can be a very convenient solution to this problem
and also brings more convenient network device management mechanism.

Now, China's Internet is divided into two parts, the national intranet and the
international network. Since 2013, China's Internet companies have had a very
significant development, the people's daily life can be satisfied with
everything, at the same time, the application of international social media
platforms China's Internet companies have also launched its replacement products.
For example, the Chinese version of Youtube is Bilibili, the Chinese version of
Google is Baidu, the Chinese version of Twitter is Weibo, the alternate to
Facebook and WhatsApp is WeChat, and the alternate to Apple Pay/Google Pay and a
series of other payment tools in China is Alipay and WeChat Pay. It can be said
that there is a Chinese app for all the daily needs of the Chinese people, which
makes most of the people's online activities only within the Chinese intranet.

\newpage
\bibliography{GFW-Brief.bib}
\begin{thebibliography}{9}
	\bibitem{DetectBlockShadowSocks} Anonymous, et al. \textit{``How China Detects
		and Blocks Shadowsocks.''} GFW Report, 29 Dec. 2019, gfw.
	report/blog/gfw\_shadowsocks. Accessed 11 Jan. 2024.

	\bibitem{GFWReport} Great Firewall Report. \textit{``Great Firewall Report.''}
	Great Firewall Repor, Dec. 2019, gfw.report. Accessed 11 Jan. 2024.

	\bibitem{CISCO}
	Schaack, Beth Van. \textit{``China's Golden Shield: Is Cisco Systems Complicit?''} Center
	for Internet and Society, 24 Mar. 2015, cyberlaw.stanford.edu/blog/2015/03/china\%E2\%80\%99s-golden-shield-cisco-systems-complicit.
	Accessed 11 Jan. 2024.

\end{thebibliography}
\end{document}
