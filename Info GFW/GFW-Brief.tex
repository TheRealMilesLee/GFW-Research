\documentclass[11pt]{article}
\title{GFW: Protection or censorship?\\ \large A brief overview on China's Great Firewall system}
\author{Hengyi Li}
\newcommand{\name}{Hengyi Li}
\usepackage[paper=letterpaper, margin=1in, headheight=13.6pt]{geometry}
\usepackage{fancyhdr}
\pagestyle{fancy}
\cfoot{Page \thepage}

\usepackage[parfill]{parskip}
\usepackage{amsmath}
\usepackage{graphicx}
\usepackage{fancyvrb}
\usepackage{upquote}
\usepackage{multicol}
\usepackage{hyperref}

\begin{document}
\maketitle
\bibliographystyle{mla}
\section{Introduction}

The Internet has come a long way today. As a bridge for the flow of information,
it carries the thoughts, ideas, opinions and needs of all people. However,
because of the value attributes of information, the flow of information may
inadvertently touch the interests of certain people or groups. This may lead to
theft by criminals, or it may be a threat to national security, but in any case,
the network firewall has been built as a result. In China, there also exists
such a wall, which is not only able to defend against cyber-attacks from outside
the country, but at the same time, and most importantly, prevents the flow of
network traffic from within the country to outside the country. This is the
subject of this article - The Great Firewall.

\section{Background}
Great Firewall (hereinafter referred to as GFW) is a traffic censorship and
attack defense system based on Cisco's intrusion detection system \cite{CISCO}.
This defense system combines active and passive detection capabilities to censor
China's Internet traffic. As far as we know from our testing, the system is
running on IPV4, but we have no way of knowing if the system is compatible with
IPV6 as well. The main reason for this is that IPV6 is not very popular in China,
and even if we could ask the carriers to enable it, they would not be very
willing to do so for security reasons, and it would require a series of very
complicated procedures to complete. What's worse, the GFW is still a black box
system for outsiders and researchers, and the existence of the system is not
recognized at the national level, and there is no official documentation about
the system. All research into the system is currently in the ``poke around and
see what happend'' phase, so most of the results and findings below come from
community organizations and web forum members. We would like to thank the
members and contributors of the gfw.repot \cite{GFWReport} website for providing a wealth of
documentation and experimental data without which we would not have been able to
gain such a deep understanding of GFW.
\newpage
\bibliography{GFW-Brief.bib}
\begin{thebibliography}{9}
\bibitem{CISCO}
Schaack, Beth Van. \textit{``China's Golden Shield: Is Cisco Systems Complicit?''} Center
for Internet and Society, 24 Mar. 2015, cyberlaw.stanford.edu/blog/2015/03/china\%E2\%80\%99s-golden-shield-cisco-systems-complicit.
Accessed 11 Jan. 2024.

\bibitem{GFWReport}
Great Firewall Report. \textit{``Great Firewall Report.''} Great Firewall Repor, Dec. 2019,
gfw.report. Accessed 11 Jan. 2024.
\end{thebibliography}
\end{document}
