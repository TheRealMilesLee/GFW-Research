\documentclass[11pt]{article}
\title{GFW: Protection or censorship?\\ \large A brief overview on China's Great Firewall system}
\author{Hengyi Li}
\newcommand{\name}{Hengyi Li}
\usepackage[paper=letterpaper, margin=1in, headheight=13.6pt]{geometry}
\usepackage{fancyhdr}
\pagestyle{fancy}
\cfoot{Page \thepage}

\usepackage[parfill]{parskip}
\usepackage{amsmath}
\usepackage{graphicx}
\usepackage{fancyvrb}
\usepackage{upquote}
\usepackage{multicol}
\usepackage{hyperref}

\begin{document}
\maketitle
\bibliographystyle{mla}
\section{Introduction}

The Internet has come a long way today. As a bridge for the flow of information,
it carries the thoughts, ideas, opinions and needs of all people. However,
because of the value attributes of information, the flow of information may
inadvertently touch the interests of certain people or groups. This may lead to
theft by criminals, or it may be a threat to national security, but in any case,
the network firewall has been built as a result. In China, there also exists
such a wall, which is not only able to defend against cyber-attacks from outside
the country, but at the same time, and most importantly, prevents the flow of
network traffic from within the country to outside the country. This is the
subject of this article - The Great Firewall.

\section{Background}


\newpage
\bibliography{GFW-Brief.bib}
\begin{thebibliography}{9}
\end{thebibliography}
\end{document}
